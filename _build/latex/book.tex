%% Generated by Sphinx.
\def\sphinxdocclass{jupyterBook}
\documentclass[letterpaper,10pt,english]{jupyterBook}
\ifdefined\pdfpxdimen
   \let\sphinxpxdimen\pdfpxdimen\else\newdimen\sphinxpxdimen
\fi \sphinxpxdimen=.75bp\relax
\ifdefined\pdfimageresolution
    \pdfimageresolution= \numexpr \dimexpr1in\relax/\sphinxpxdimen\relax
\fi
%% let collapsible pdf bookmarks panel have high depth per default
\PassOptionsToPackage{bookmarksdepth=5}{hyperref}
%% turn off hyperref patch of \index as sphinx.xdy xindy module takes care of
%% suitable \hyperpage mark-up, working around hyperref-xindy incompatibility
\PassOptionsToPackage{hyperindex=false}{hyperref}
%% memoir class requires extra handling
\makeatletter\@ifclassloaded{memoir}
{\ifdefined\memhyperindexfalse\memhyperindexfalse\fi}{}\makeatother

\PassOptionsToPackage{booktabs}{sphinx}
\PassOptionsToPackage{colorrows}{sphinx}

\PassOptionsToPackage{warn}{textcomp}

\catcode`^^^^00a0\active\protected\def^^^^00a0{\leavevmode\nobreak\ }
\usepackage{cmap}
\usepackage{fontspec}
\defaultfontfeatures[\rmfamily,\sffamily,\ttfamily]{}
\usepackage{amsmath,amssymb,amstext}
\usepackage{polyglossia}
\setmainlanguage{english}



\setmainfont{FreeSerif}[
  Extension      = .otf,
  UprightFont    = *,
  ItalicFont     = *Italic,
  BoldFont       = *Bold,
  BoldItalicFont = *BoldItalic
]
\setsansfont{FreeSans}[
  Extension      = .otf,
  UprightFont    = *,
  ItalicFont     = *Oblique,
  BoldFont       = *Bold,
  BoldItalicFont = *BoldOblique,
]
\setmonofont{FreeMono}[
  Extension      = .otf,
  UprightFont    = *,
  ItalicFont     = *Oblique,
  BoldFont       = *Bold,
  BoldItalicFont = *BoldOblique,
]



\usepackage[Bjarne]{fncychap}
\usepackage[,numfigreset=1,mathnumfig]{sphinx}

\fvset{fontsize=\small}
\usepackage{geometry}


% Include hyperref last.
\usepackage{hyperref}
% Fix anchor placement for figures with captions.
\usepackage{hypcap}% it must be loaded after hyperref.
% Set up styles of URL: it should be placed after hyperref.
\urlstyle{same}


\usepackage{sphinxmessages}


\usepackage{graphicx}
% Scale any image wider than \linewidth down to 90 %
\setkeys{Gin}{width=\linewidth,keepaspectratio}

        % Start of preamble defined in sphinx-jupyterbook-latex %
         \usepackage[Latin,Greek]{ucharclasses}
        \usepackage{unicode-math}
        % fixing title of the toc
        \addto\captionsenglish{\renewcommand{\contentsname}{Contents}}
        \hypersetup{
            pdfencoding=auto,
            psdextra
        }
        % End of preamble defined in sphinx-jupyterbook-latex %
        

\title{Fiji Lab Walkthrough}
\date{Apr 30, 2025}
\release{}
\author{The Fiji Community}
\newcommand{\sphinxlogo}{\vbox{}}
\renewcommand{\releasename}{}
\makeindex
\begin{document}

\pagestyle{empty}
\sphinxmaketitle
\pagestyle{plain}
\sphinxtableofcontents
\pagestyle{normal}
\phantomsection\label{\detokenize{intro::doc}}


\sphinxAtStartPar
Information courtesy of Helen Wilson, Michael Nelson, Edward Evans, Ellen Dobson, Curtis Reuden,
as well as well as many others involved in ImageJ/FIJI development and production of
educational materials, walkthroughs, and demonstrations.
\begin{itemize}
\item {} 
\sphinxAtStartPar
{\hyperref[\detokenize{installation::doc}]{\sphinxcrossref{Installing ImageJ/FIJI}}}

\item {} 
\sphinxAtStartPar
{\hyperref[\detokenize{keyboard-shortcuts::doc}]{\sphinxcrossref{Common Keyboard Shortcuts}}}

\item {} 
\sphinxAtStartPar
{\hyperref[\detokenize{basics::doc}]{\sphinxcrossref{Getting Started, Opening an Image}}}

\item {} 
\sphinxAtStartPar
{\hyperref[\detokenize{basic-segmentation::doc}]{\sphinxcrossref{Basic Segmentation}}}

\item {} 
\sphinxAtStartPar
{\hyperref[\detokenize{basic-registration::doc}]{\sphinxcrossref{Basic Registration}}}

\item {} 
\sphinxAtStartPar
{\hyperref[\detokenize{basic-scripting::doc}]{\sphinxcrossref{Scripting}}}

\item {} 
\sphinxAtStartPar
{\hyperref[\detokenize{deconvolution::doc}]{\sphinxcrossref{Deconvolution}}}

\item {} 
\sphinxAtStartPar
{\hyperref[\detokenize{background-subtraction::doc}]{\sphinxcrossref{Background Subtraction – Gaussian Filters}}}

\item {} 
\sphinxAtStartPar
{\hyperref[\detokenize{trackmate::doc}]{\sphinxcrossref{TrackMate – Example from documentation}}}

\item {} 
\sphinxAtStartPar
{\hyperref[\detokenize{labkit::doc}]{\sphinxcrossref{Labkit Segmentation}}}

\item {} 
\sphinxAtStartPar
{\hyperref[\detokenize{extra-resources::doc}]{\sphinxcrossref{Other Resources}}}

\end{itemize}

\sphinxstepscope


\chapter{Installing ImageJ/FIJI}
\label{\detokenize{installation:installing-imagej-fiji}}\label{\detokenize{installation::doc}}

\section{Download FIJI Package}
\label{\detokenize{installation:download-fiji-package}}\begin{enumerate}
\sphinxsetlistlabels{\arabic}{enumi}{enumii}{}{.}%
\item {} 
\sphinxAtStartPar
Download FIJI package from the software site
FIJI has pre\sphinxhyphen{}installed plugins and is recommended for this lab.
Download it at \sphinxurl{https://imagej.net/software/fiji/downloads}.
Look for the zip folder and extract the contents to the desired location on your computer. Keeping it in “Program Files” is generally not recommended to avoid access/security issues.

\end{enumerate}


\section{Installation}
\label{\detokenize{installation:installation}}
\sphinxAtStartPar
We’re going to run the new Jaunch launcher, because it runs faster and allows running without security popups. If it doesn’t show up within two minutes of running, then run the backup launcher.


\subsection{Windows}
\label{\detokenize{installation:windows}}\begin{itemize}
\item {} 
\sphinxAtStartPar
Right\sphinxhyphen{}click on \sphinxcode{\sphinxupquote{fiji\sphinxhyphen{}win64.zip}} and choose \sphinxstylestrong{Extract All…} to \sphinxcode{\sphinxupquote{Downloads\textbackslash{}}} (avoid placing it in \sphinxcode{\sphinxupquote{Program Files}}).

\item {} 
\sphinxAtStartPar
Open the \sphinxcode{\sphinxupquote{Fiji64.app}} folder.

\item {} 
\sphinxAtStartPar
\sphinxstylestrong{New launcher}: Run  \sphinxcode{\sphinxupquote{Fiji\sphinxhyphen{}windows\sphinxhyphen{}x64.exe}} in that folder.

\item {} 
\sphinxAtStartPar
Running for the first time may take about 60 seconds.

\item {} 
\sphinxAtStartPar
\sphinxstylestrong{Backup launcher}: Run \sphinxcode{\sphinxupquote{ImageJ\sphinxhyphen{}win64.exe}} in that folder.

\end{itemize}


\subsection{Linux}
\label{\detokenize{installation:linux}}\begin{itemize}
\item {} 
\sphinxAtStartPar
Unzip \sphinxcode{\sphinxupquote{fiji\sphinxhyphen{}linux64.zip}}:

\begin{sphinxVerbatim}[commandchars=\\\{\}]
unzip\PYG{+w}{ }fiji\PYGZhy{}linux64.zip
\PYG{n+nb}{cd}\PYG{+w}{ }Fiji.app/
\end{sphinxVerbatim}

\item {} 
\sphinxAtStartPar
\sphinxstylestrong{New launcher}:

\begin{sphinxVerbatim}[commandchars=\\\{\}]
./fiji\PYGZhy{}linux\PYGZhy{}x64
\end{sphinxVerbatim}

\item {} 
\sphinxAtStartPar
\sphinxstylestrong{Backup launcher}:

\begin{sphinxVerbatim}[commandchars=\\\{\}]
./ImageJ\PYGZhy{}linux64
\end{sphinxVerbatim}

\end{itemize}


\subsection{macOS}
\label{\detokenize{installation:macos}}\begin{itemize}
\item {} 
\sphinxAtStartPar
Double\sphinxhyphen{}click \sphinxcode{\sphinxupquote{fiji\sphinxhyphen{}macos.zip}} to extract.

\item {} 
\sphinxAtStartPar
(Optional: copy \sphinxcode{\sphinxupquote{Fiji.app}} to Desktop.)

\item {} 
\sphinxAtStartPar
Right\sphinxhyphen{}click on  \sphinxcode{\sphinxupquote{Fiji.app}}, choose \sphinxstylestrong{Show Package Contents}.

\item {} 
\sphinxAtStartPar
You will see a folder containing another  \sphinxcode{\sphinxupquote{Fiji}} icon inside it.\\
Double\sphinxhyphen{}click the \sphinxstylestrong{second} \sphinxcode{\sphinxupquote{Fiji}} icon to launch.

\item {} 
\sphinxAtStartPar
If it doesn’t work, see backup instructions below.

\end{itemize}


\subsubsection{Backup Instructions for macOS}
\label{\detokenize{installation:backup-instructions-for-macos}}\begin{itemize}
\item {} 
\sphinxAtStartPar
Double\sphinxhyphen{}click \sphinxcode{\sphinxupquote{fiji\sphinxhyphen{}macos.zip}} to extract.

\item {} 
\sphinxAtStartPar
(Optional: copy \sphinxcode{\sphinxupquote{Fiji.app}} to Desktop.)

\item {} 
\sphinxAtStartPar
Try opening  \sphinxcode{\sphinxupquote{Fiji.app}}.
This will likely be rejected by macOS for security reasons.

\item {} 
\sphinxAtStartPar
Open \sphinxstylestrong{Apple Menu > System Settings}.

\item {} 
\sphinxAtStartPar
Search for \sphinxstylestrong{Gatekeeper}.

\item {} 
\sphinxAtStartPar
You should see the text\\
“Fiji.app was blocked…”\_ → click \sphinxstylestrong{Open Anyway}.

\item {} 
\sphinxAtStartPar
Enter your password or use your fingerprint if prompted.

\item {} 
\sphinxAtStartPar
Click \sphinxstylestrong{Open}.\\
After this, you should be able to open \sphinxcode{\sphinxupquote{Fiji.app}} directly.

\end{itemize}


\subsection{A Side Note on Updating}
\label{\detokenize{installation:a-side-note-on-updating}}
\sphinxAtStartPar
When using FIJI, it is recommended to use \sphinxstylestrong{Help > Update} rather than \sphinxstylestrong{Help > Update ImageJ}.\\
Update ImageJ will specifically update only the base ImageJ package within FIJI, not the associated plugins.\\
This makes it more likely that you will run into version issues and other update problems.\\
Generally, updating when prompted upon opening FIJI will also help prevent issues.


\section{Install Plugins with Update Sites}
\label{\detokenize{installation:install-plugins-with-update-sites}}\begin{enumerate}
\sphinxsetlistlabels{\arabic}{enumi}{enumii}{}{.}%
\item {} 
\sphinxAtStartPar
After running \sphinxstylestrong{Help > Update}, there will be an option to \sphinxstylestrong{Manage Update Sites}.\\
This provides access to common plugins that are compatible with FIJI. Use the checkbox to add a plugin, then click \sphinxstylestrong{Apply and Close}.

\end{enumerate}

\sphinxAtStartPar
\sphinxincludegraphics{{update-fiji}.png}
\begin{enumerate}
\sphinxsetlistlabels{\arabic}{enumi}{enumii}{}{.}%
\setcounter{enumi}{1}
\item {} 
\sphinxAtStartPar
Click \sphinxstylestrong{Apply Changes}, then restart FIJI.

\item {} 
\sphinxAtStartPar
You should now be able to search for and run the plugin.

\end{enumerate}


\section{Downloading a Practice Dataset}
\label{\detokenize{installation:downloading-a-practice-dataset}}
\sphinxAtStartPar
The complete list of files may be downloaded at the following links:

\sphinxAtStartPar
Google Drive:
\sphinxurl{https://drive.google.com/drive/folders/1lgn-S5fZZZX0mwou23S\_f6d3dG2Di64g?usp=drive\_link}

\sphinxAtStartPar
Github:
\sphinxhref{https://github.com/brp-optics/FIJI\_Lab\_Tutorial}{https://github.com/brp\sphinxhyphen{}optics/FIJI\_Lab\_Tutoriall}

\sphinxstepscope


\chapter{Common Keyboard Shortcuts}
\label{\detokenize{keyboard-shortcuts:common-keyboard-shortcuts}}\label{\detokenize{keyboard-shortcuts::doc}}

\begin{savenotes}\sphinxattablestart
\sphinxthistablewithglobalstyle
\centering
\begin{tabulary}{\linewidth}[t]{TT}
\sphinxtoprule
\sphinxstyletheadfamily 
\sphinxAtStartPar
Shortcut
&\sphinxstyletheadfamily 
\sphinxAtStartPar
Action
\\
\sphinxmidrule
\sphinxtableatstartofbodyhook
\sphinxAtStartPar
\sphinxcode{\sphinxupquote{+}}
&
\sphinxAtStartPar
Zoom in
\\
\sphinxhline
\sphinxAtStartPar
\sphinxcode{\sphinxupquote{\sphinxhyphen{}}}
&
\sphinxAtStartPar
Zoom out
\\
\sphinxhline
\sphinxAtStartPar
\sphinxcode{\sphinxupquote{Ctrl/Command + L}}
&
\sphinxAtStartPar
Focus search bar
\\
\sphinxhline
\sphinxAtStartPar
\sphinxcode{\sphinxupquote{Ctrl/Command + S}}
&
\sphinxAtStartPar
Save
\\
\sphinxhline
\sphinxAtStartPar
\sphinxstyleemphasis{(No shortcut)}
&
\sphinxAtStartPar
Save As
\\
\sphinxhline
\sphinxAtStartPar
\sphinxcode{\sphinxupquote{Ctrl/Command + Shift + C}}
&
\sphinxAtStartPar
Adjust Brightness and Contrast
\\
\sphinxhline
\sphinxAtStartPar
\sphinxcode{\sphinxupquote{Ctrl/Command + Shift + Z}}
&
\sphinxAtStartPar
Color channel control
\\
\sphinxhline
\sphinxAtStartPar
\sphinxcode{\sphinxupquote{Ctrl/Command + Z}}
&
\sphinxAtStartPar
Undo
\\
\sphinxhline
\sphinxAtStartPar
\sphinxcode{\sphinxupquote{Ctrl + H}} (Windows)  \sphinxcode{\sphinxupquote{H}} (Mac)
&
\sphinxAtStartPar
Open Histogram
\\
\sphinxhline
\sphinxAtStartPar
\sphinxcode{\sphinxupquote{Ctrl/Command + K}}
&
\sphinxAtStartPar
Plot / Line Profile
\\
\sphinxhline
\sphinxAtStartPar
\sphinxcode{\sphinxupquote{Ctrl/Command + M}}
&
\sphinxAtStartPar
Measure
\\
\sphinxhline
\sphinxAtStartPar
\sphinxcode{\sphinxupquote{Ctrl/Command + Shift + T}}
&
\sphinxAtStartPar
Threshold
\\
\sphinxhline
\sphinxAtStartPar
\sphinxcode{\sphinxupquote{Shift}}
&
\sphinxAtStartPar
Draw straight lines
\\
\sphinxhline
\sphinxAtStartPar
\sphinxcode{\sphinxupquote{Alt + Arrow Keys}}
&
\sphinxAtStartPar
Change current selection size
\\
\sphinxhline
\sphinxAtStartPar
\sphinxcode{\sphinxupquote{T}}
&
\sphinxAtStartPar
Open ROI (Region of Interest) Manager and add current selection as ROI
\\
\sphinxhline
\sphinxAtStartPar
\sphinxcode{\sphinxupquote{Ctrl/Command + Shift + W}}
&
\sphinxAtStartPar
Close all windows
\\
\sphinxbottomrule
\end{tabulary}
\sphinxtableafterendhook\par
\sphinxattableend\end{savenotes}


\section{Also Helpful}
\label{\detokenize{keyboard-shortcuts:also-helpful}}\begin{itemize}
\item {} 
\sphinxAtStartPar
\sphinxstylestrong{Window > Tile}\\
Montages all open images/windows.

\item {} 
\sphinxAtStartPar
\sphinxstylestrong{Image > Transform}\\
Flip or rotate images.

\item {} 
\sphinxAtStartPar
\sphinxstylestrong{Help > Update}\\
Update FIJI.

\item {} 
\sphinxAtStartPar
\sphinxstylestrong{Enter} (when focused on an image)\\
Surfaces the ImageJ main toolbar window.

\end{itemize}

\sphinxstepscope


\chapter{Getting Started, Opening an Image}
\label{\detokenize{basics:getting-started-opening-an-image}}\label{\detokenize{basics::doc}}
\sphinxAtStartPar
The easiest way to open a data set is to click\sphinxhyphen{}and\sphinxhyphen{}drag from the folder. However, some file types
need to be specifically imported. To import a sequence of images that are saved within a folder,
you can use the \sphinxstylestrong{File > Import > Image Sequence} option to open them as a stack or as
separate images.
Additionally, using a \sphinxstylestrong{Virtual Stack} in the \sphinxstylestrong{Bio\sphinxhyphen{}Formats Importer} may be useful if trying to view
large data files. This makes it easier to get a quick look at data when there is limited RAM.
However, if you need to perform an image operation, such as background subtraction, then
more memory will be required as the data needs to be loaded.

\begin{sphinxadmonition}{note}{Note:}
\sphinxAtStartPar
💡 \sphinxstylestrong{Pro tip}: Press \sphinxcode{\sphinxupquote{Ctrl + L}} (Windows) or \sphinxcode{\sphinxupquote{Command + L}} (Mac) to jump straight to the search bar.\\
This can access nearly any menu command or plugin, usually faster than using the menus!
\end{sphinxadmonition}


\section{Starting with a basic image}
\label{\detokenize{basics:starting-with-a-basic-image}}
\sphinxAtStartPar
Open “blobs.gif”, via \sphinxstylestrong{File > Open Samples > Blobs}, or \sphinxstylestrong{Ctrl + Shift + B}


\section{Inspecting Images}
\label{\detokenize{basics:inspecting-images}}\begin{enumerate}
\sphinxsetlistlabels{\arabic}{enumi}{enumii}{}{.}%
\item {} 
\sphinxAtStartPar
Hover over an area in the image.\\
In the status bar of Fiji, you should see the pixel you are on, with XY coordinates, and its associated value.
\begin{itemize}
\item {} 
\sphinxAtStartPar
In an RGB type image, Fiji will also tell you the associated color values for the
red, green, and blue channels.

\end{itemize}

\end{enumerate}

\sphinxAtStartPar
\sphinxincludegraphics{{status-bar}.png}
\begin{enumerate}
\sphinxsetlistlabels{\arabic}{enumi}{enumii}{}{.}%
\setcounter{enumi}{1}
\item {} 
\sphinxAtStartPar
To see a larger area, use the \sphinxstylestrong{Pixel Inspection Tool}:
\begin{itemize}
\item {} 
\sphinxAtStartPar
In the Fiji window, click the Pixel Inspection icon.

\item {} 
\sphinxAtStartPar
\sphinxstyleemphasis{If the icon is not present, you can use the double red arrows to add the icon, by
opening the list and selecting \sphinxstylestrong{Pixel Inspector}.}

\end{itemize}

\end{enumerate}

\sphinxAtStartPar
\sphinxincludegraphics{{pixel-inspector}.png}
\begin{itemize}
\item {} 
\sphinxAtStartPar
This brings up a window of pixel values that can be moved around the image.\\
To adjust the parameters of this window, click \sphinxstylestrong{Prefs}.\\
This can be used to adjust the size of the window.

\end{itemize}

\sphinxAtStartPar
\sphinxincludegraphics{{pixel-values}.png}


\bigskip\hrule\bigskip



\section{Inverting the Background}
\label{\detokenize{basics:inverting-the-background}}\begin{itemize}
\item {} 
\sphinxAtStartPar
Sometimes the background may be white instead of black, depending on what you are
interested in within the image.

\item {} 
\sphinxAtStartPar
The color scheme can be inverted by selecting \sphinxstylestrong{LUT > Invert LUT}.
\begin{quote}

\sphinxAtStartPar
\sphinxstyleemphasis{Note}: if you use \sphinxstylestrong{Edit > Invert}, this will invert the actual image and change
the pixel values themselves, which may not represent the original data.
\end{quote}

\end{itemize}


\section{Changing the Color Scheme (LUT)}
\label{\detokenize{basics:changing-the-color-scheme-lut}}
\sphinxAtStartPar
A Lookup Table (LUT) controls the colormap of the images.
This can easily be changed using the \sphinxstylestrong{LUT} button in the toolbar.
Try a few LUTs and see how the information in the background and foreground changes.
\begin{enumerate}
\sphinxsetlistlabels{\arabic}{enumi}{enumii}{}{.}%
\item {} 
\sphinxAtStartPar
Using \sphinxstylestrong{Image > Colors > Display LUTs} will display different types of LUTs.

\item {} 
\sphinxAtStartPar
It is also helpful to include a scale/calibration bar when changing LUTs.\\
This can be added with \sphinxstylestrong{Analyze > Tools > Calibration Bar}.

\end{enumerate}

\sphinxAtStartPar
\sphinxincludegraphics{{blobs-luts}.png}


\subsection{Colorblindness Consideration}
\label{\detokenize{basics:colorblindness-consideration}}
\sphinxAtStartPar
To test how different images may appear to those with colorblindness, you can use the \sphinxstylestrong{Simulate Color Blindness} plugin if you have an RGB image.\\
\sphinxcode{\sphinxupquote{mpl\sphinxhyphen{}viridis}} can be a good option as it is designed to be \sphinxstyleemphasis{perceptually uniform} (see \sphinxurl{https://imagej.net/imaging/visualization}).

\sphinxAtStartPar
As a separate note, it may also be helpful to test any publication figures (images or
otherwise) in grayscale, to see what color information may be lost if the paper is printed in
grayscale.\\
In many cases, it may be more practical to develop images/figures in grayscale,
especially for presentations, and it is still a valid representation of the data.


\section{Plot/Line Plot}
\label{\detokenize{basics:plot-line-plot}}\begin{enumerate}
\sphinxsetlistlabels{\arabic}{enumi}{enumii}{}{.}%
\item {} 
\sphinxAtStartPar
To see graphically how the pixel values vary across the image we can use the plot tool.
\begin{itemize}
\item {} 
\sphinxAtStartPar
This can be very useful for determining how much noise is in the background of an
image, or for finding the width of an object, such as a point spread function.

\end{itemize}

\item {} 
\sphinxAtStartPar
Draw a line or rectangle across the image, using the tools from the toolbar:

\end{enumerate}

\sphinxAtStartPar
\sphinxincludegraphics{{line-tool}.png}
\begin{enumerate}
\sphinxsetlistlabels{\arabic}{enumi}{enumii}{}{.}%
\setcounter{enumi}{2}
\item {} 
\sphinxAtStartPar
Press \sphinxstylestrong{Ctrl + K} to display the profile along the line.
\begin{itemize}
\item {} 
\sphinxAtStartPar
The \sphinxstylestrong{Live} button in the plot window can be used to create an active plot that
changes when the selection in the image is changed.

\end{itemize}

\item {} 
\sphinxAtStartPar
Here we can clearly see the increase in signal intensity across the three selected
blobs.\\
Since this is a rectangular selection, the values are averaged across the height of the rectangle.

\end{enumerate}

\sphinxAtStartPar
\sphinxincludegraphics{{plot-profile}.png}
\begin{enumerate}
\sphinxsetlistlabels{\arabic}{enumi}{enumii}{}{.}%
\setcounter{enumi}{4}
\item {} 
\sphinxAtStartPar
In comparison, the plot of a single line shows similar intensity changes when
carefully drawn through the three blobs, but the plot is noisier and more sensitive
to where the line is placed.

\end{enumerate}

\sphinxAtStartPar
\sphinxincludegraphics{{plot-profile-line}.png}


\section{Opening and viewing a Z\sphinxhyphen{}Stack}
\label{\detokenize{basics:opening-and-viewing-a-z-stack}}\begin{enumerate}
\sphinxsetlistlabels{\arabic}{enumi}{enumii}{}{.}%
\item {} 
\sphinxAtStartPar
Open “t1\sphinxhyphen{}head.tif”, via \sphinxstylestrong{File > Open Samples > T1 Head (16\sphinxhyphen{}bits)}.

\item {} 
\sphinxAtStartPar
Use the bar to scroll through the stack of images

\item {} 
\sphinxAtStartPar
We can also display the orthogonal view by using \sphinxstylestrong{Image > Stacks > Orthogonal Views}
or \sphinxstylestrong{Ctrl/Command + Shift + H}

\end{enumerate}
\begin{itemize}
\item {} 
\sphinxAtStartPar
The yellow crosshairs can be used to change the display for each orthogonal
view, but one has to close the orthogonal view to select images.

\end{itemize}
\begin{enumerate}
\sphinxsetlistlabels{\arabic}{enumi}{enumii}{}{.}%
\setcounter{enumi}{3}
\item {} 
\sphinxAtStartPar
To select from an orthogonal view, try \sphinxstylestrong{reslice}, in \sphinxstylestrong{Image > Stacks > Reslice} or press \sphinxstylestrong{\textbackslash{} (backslash)} when the image is in focus.

\end{enumerate}




\section{Hyperstacks}
\label{\detokenize{basics:hyperstacks}}
\sphinxAtStartPar
Hyperstacks are multidimensional data, such as a multichannel timeseries or z\sphinxhyphen{}stacks.
\begin{enumerate}
\sphinxsetlistlabels{\arabic}{enumi}{enumii}{}{.}%
\item {} 
\sphinxAtStartPar
Start by opening the “hela\_infected\_timeseries.tif” image. This is available for download
at: \sphinxurl{https://media.imagej.net/workshops/data/3d/}

\end{enumerate}
\begin{itemize}
\item {} 
\sphinxAtStartPar
If internet access is unavailable, the \sphinxcode{\sphinxupquote{Mitosis (5D stack)}} sample from \sphinxstylestrong{File > Open Samples} works, too.

\end{itemize}


\subsection{Duplicate A Channel Or Image}
\label{\detokenize{basics:duplicate-a-channel-or-image}}\begin{itemize}
\item {} 
\sphinxAtStartPar
Use \sphinxstylestrong{Image > Duplicate} to bring up the duplication options window. This can also be
done with \sphinxstylestrong{Ctrl + Shift + D} or \sphinxstylestrong{Right Click > Duplicate}. Here, we can specify which RGB
channels and timepoints we want to separate. If we use 1\sphinxhyphen{}2 in the channels menu, it
should produce an image of just the red and green channels. The duplicate hyperstack box should be checked.
(If you are using the mitosis image, try choosing just one value for the color channel.)

\end{itemize}

\sphinxAtStartPar
\sphinxincludegraphics{{duplicate}.png}
\sphinxincludegraphics{{hela-duplicate}.png}


\section{Changing Channel Colors}
\label{\detokenize{basics:changing-channel-colors}}\begin{enumerate}
\sphinxsetlistlabels{\arabic}{enumi}{enumii}{}{.}%
\item {} 
\sphinxAtStartPar
To change the display color of channels, use \sphinxstylestrong{Image > Color > Channels Tool} or \sphinxstylestrong{Ctrl + Shift + Z}.

\end{enumerate}

\sphinxAtStartPar
\sphinxincludegraphics{{channels}.png}

\sphinxAtStartPar
Note: The “more” option can be used to apply colors to the selected channels.


\section{3D Volume and Projections}
\label{\detokenize{basics:d-volume-and-projections}}\begin{enumerate}
\sphinxsetlistlabels{\arabic}{enumi}{enumii}{}{.}%
\item {} 
\sphinxAtStartPar
Open \sphinxcode{\sphinxupquote{t1\sphinxhyphen{}head.tif}}, \sphinxstylestrong{File > Open Samples > T1 Head (16 bits)}

\item {} 
\sphinxAtStartPar
To display the 3D Volume, use \sphinxstylestrong{Image > Stack > 3D Projection > Click ok}
In the 3D projection dialogue box, there are various options for projection including
axis of rotation and setting the slice spacing (step size) which may be useful for
known parameters.

\end{enumerate}

\sphinxAtStartPar
\sphinxincludegraphics{{t1-head-max-proj}.png}
\begin{enumerate}
\sphinxsetlistlabels{\arabic}{enumi}{enumii}{}{.}%
\setcounter{enumi}{2}
\item {} 
\sphinxAtStartPar
Additionally, the original image stack may be projected along one axis using \sphinxstylestrong{Image >
Stacks > Z Project}. It is important to note that the information in the projection is highly
dependent on the display parameter. A few examples are shown below (left to right:
single slice of the stack, average projection, median projection, max projection).

\end{enumerate}

\sphinxAtStartPar
\sphinxincludegraphics{{t-head-multiple-z-projections}.png}


\section{Changing Image Types, cautiously.}
\label{\detokenize{basics:changing-image-types-cautiously}}
\sphinxAtStartPar
Some operations \sphinxhyphen{} especially image math operations \sphinxhyphen{} only work on images of a certain type, and sometimes we desire images of a certain type. For example, if we want to multiply two 8\sphinxhyphen{}bit images, it would be prudent to first convert them to 16\sphinxhyphen{}bit images so the output is to a 16\sphinxhyphen{}bit image and does not overflow.

\sphinxAtStartPar
However, changing the image type through \sphinxstylestrong{Image >
Type} can result in the loss of information as we will see in the next example. Not all image
types can be converted back to the original. The following error is likely to display if you
are not able to complete a type conversion.

\sphinxAtStartPar
\sphinxincludegraphics{{type-converter-error}.png}


\section{Bit Depth}
\label{\detokenize{basics:bit-depth}}\begin{enumerate}
\sphinxsetlistlabels{\arabic}{enumi}{enumii}{}{.}%
\item {} 
\sphinxAtStartPar
Open the \sphinxcode{\sphinxupquote{m51.tif}} image, \sphinxstylestrong{Open > Open Samples > M51 Galaxy (16 Bits)}.

\item {} 
\sphinxAtStartPar
Adjust the contrast so we can see the image a bit better. \sphinxstylestrong{Image > Adjust > Brightness and Contrast}.

\item {} 
\sphinxAtStartPar
Make a histogram of the image, \sphinxstylestrong{Ctrl + H}, and notice the scale of the histogram and the
mean value. Keep this window open.
a. 16\sphinxhyphen{}bit images have a maximum value of 65,536.

\item {} 
\sphinxAtStartPar
Convert the 16 bit image to an 8 bit image, \sphinxstylestrong{Image > Type > 8 bit}.
a. 8\sphinxhyphen{}bit images have a maximum of 256 values. This means the data resolution and
range is lower compared to a higher bit depth, so the data is compressed.

\item {} 
\sphinxAtStartPar
Make another histogram and keep the window open. How did the values change?

\end{enumerate}

\sphinxAtStartPar
\sphinxincludegraphics{{m51-bit-depth}.png}


\section{Adjusting Brightness and Contrast}
\label{\detokenize{basics:adjusting-brightness-and-contrast}}\begin{itemize}
\item {} 
\sphinxAtStartPar
Changing the contrast in an image can be a good visualization tool. Using \sphinxstylestrong{Ctrl + Shift +
C} or \sphinxstylestrong{Image > Adjust > Brightness and Contrast} will display the B\&C menu. Using
Auto or the B\&C sliders changes the look up table (LUT), which does not change the pixel values. This is good for visualization, but \sphinxstylestrong{using the Apply button will
change the pixel values, which generally should not be done.}

\end{itemize}


\section{Set Scale and Scale Bars}
\label{\detokenize{basics:set-scale-and-scale-bars}}\begin{enumerate}
\sphinxsetlistlabels{\arabic}{enumi}{enumii}{}{.}%
\item {} 
\sphinxAtStartPar
If the size per pixel is known, such as from a microscope calibration, this can be
used to change the image dimensions from pixels to the known field of view.

\item {} 
\sphinxAtStartPar
Use \sphinxstylestrong{Analyze > Set Scale} to enter the known distance.
\begin{itemize}
\item {} 
\sphinxAtStartPar
For example, an image with a pixel size of \sphinxcode{\sphinxupquote{0.73 pixels/micron}} would be entered as shown, using “um” or “micron”.

\end{itemize}

\end{enumerate}

\sphinxAtStartPar
\sphinxincludegraphics{{set-scale}.png}
\begin{itemize}
\item {} 
\sphinxAtStartPar
Further info for setting the scale based on a measurement in an image can be found here:\\
\sphinxurl{https://serc.carleton.edu/eyesinthesky2/week2/get\_to\_know\_imagej.html}

\end{itemize}
\begin{enumerate}
\sphinxsetlistlabels{\arabic}{enumi}{enumii}{}{.}%
\setcounter{enumi}{2}
\item {} 
\sphinxAtStartPar
To add a scale bar, use \sphinxstylestrong{Analyze > Tools > Scale Bar}.

\end{enumerate}

\sphinxAtStartPar
\sphinxincludegraphics{{scale-bar}.png}


\section{Accessing Image Metadata}
\label{\detokenize{basics:accessing-image-metadata}}


\sphinxstepscope


\chapter{Basic Segmentation}
\label{\detokenize{basic-segmentation:basic-segmentation}}\label{\detokenize{basic-segmentation::doc}}
\sphinxAtStartPar
For this section, we demonstrate with the \sphinxstylestrong{embryos} sample.


\section{Background Subtraction}
\label{\detokenize{basic-segmentation:background-subtraction}}
\sphinxAtStartPar
Oftentimes the background illumination is not uniform, and this interferes with intensity\sphinxhyphen{}based thresholding. Fiji includes multiple methods to perform background subtraction in ImageJ, including \sphinxstylestrong{Process > Subtract Background}, \sphinxstylestrong{Mean/Median Background Subtraction}, or \sphinxstylestrong{Gaussian Blur Subtraction}. If you are not planning a quantitative analysis, \sphinxstylestrong{Enhance Local Contrast} (\sphinxstylestrong{CLAHE}, Contrast Limited Adaptive Histogram Equalization) may also be worth a try.

\sphinxAtStartPar
\sphinxincludegraphics{{embryos-bg}.png}
\sphinxstyleemphasis{Left to right: original, background subtracted, with CLAHE.}

\sphinxAtStartPar
However, any background subtraction used to process images should be carefully considered in relation to the original source of noise, how the image information is used, and any further processing or quantification of the image.

\sphinxAtStartPar
In some cases the Subtract Background plugin, or other methods, can also introduce artifacts that are not present in the original image.

\sphinxAtStartPar
Some \sphinxhref{http://Image.sc}{Image.sc} forum discussion here:\\
\sphinxurl{https://forum.image.sc/t/consensus-on-subtract-background-built-in-or-other/7061}


\bigskip\hrule\bigskip



\section{Segmentation / Thresholding}
\label{\detokenize{basic-segmentation:segmentation-thresholding}}
\sphinxAtStartPar
For this section we will open up blobs again: \sphinxstylestrong{Ctrl\sphinxhyphen{}Shift\sphinxhyphen{}B}
\begin{enumerate}
\sphinxsetlistlabels{\arabic}{enumi}{enumii}{}{.}%
\item {} 
\sphinxAtStartPar
If we want to segment out the blobs we are interested in, one possible method is to use an intensity threshold. First, let’s look at the histogram of the intensity values using \sphinxstylestrong{Ctrl + H}.

\end{enumerate}

\sphinxAtStartPar
\sphinxincludegraphics{{blobs-hist}.png}

\sphinxAtStartPar
We can see that most of the background pixels have an intensity value around 50, whereas there is another grouping of pixel intensities around 210, which is likely the areas we are interested in because this image has a white background. Notice that the values are stretched outwards; this indicates that the original range was likely very small or somehow adjusted.
\begin{enumerate}
\sphinxsetlistlabels{\arabic}{enumi}{enumii}{}{.}%
\setcounter{enumi}{1}
\item {} 
\sphinxAtStartPar
Use \sphinxstylestrong{Image > Adjust > Threshold} to see the automatic threshold applied based on the original histogram.\\
This seems to segment the blobs pretty well. You can change the slider bars to see how the segmentation changes as the threshold moves.\\
When ready, click \sphinxstylestrong{Apply} to create the binarized mask. Notice that the blobs now have a value of \sphinxcode{\sphinxupquote{255}}, and the background has been set to \sphinxcode{\sphinxupquote{0}}.  This represents a \sphinxstylestrong{permanent change} in the image values, so going forward, most quantified analysis involving pixel values must be performed by applying the segmentation as a mask on the original image (see {[}Masks{]} section).

\end{enumerate}

\sphinxAtStartPar
\sphinxincludegraphics{{blobs-thresh}.png}


\begin{itemize}
\item {} 
\sphinxAtStartPar
There is also an \sphinxstylestrong{Auto Threshold} option under \sphinxstylestrong{Image > Adjust > Auto Threshold}, that you can use to compare different methods of thresholding. It is easiest to see the comparisons using the \sphinxstylestrong{Try All} method.

\end{itemize}

\sphinxAtStartPar
\sphinxincludegraphics{{blobs-autothresh}.png}


\bigskip\hrule\bigskip



\section{Morphological filtering}
\label{\detokenize{basic-segmentation:morphological-filtering}}
\sphinxAtStartPar
Often the mask of the particles will pick up a lot of noise, and it is useful to filter them. The morphological operations \sphinxstylestrong{Erode} and \sphinxstylestrong{Dilate} from the menu \sphinxstylestrong{Process > Binary} can be combined in sequence to remove noise. When Erode is run first, the sequence is called \sphinxstylestrong{Open}, and removes small objects and speckle noise. When Dilate is run first, the sequence is called \sphinxstylestrong{Close} and fills small holes or gaps between mask objects.

\sphinxAtStartPar
\sphinxincludegraphics{{blobs-morpho}.png}


\bigskip\hrule\bigskip



\section{Splitting chunks by watershed}
\label{\detokenize{basic-segmentation:splitting-chunks-by-watershed}}
\sphinxAtStartPar
Often two smooth convex objects (cells!) will be touching and thereby detected as part of the same mask element. \sphinxstylestrong{Watershed segmentation} or \sphinxstylestrong{Process > Binary > Watershed} splits masks by drawing a narrow channel at narrow bottlenecks. This is useful for separating cells that were identified as part of the same mask.

\sphinxAtStartPar
\sphinxincludegraphics{{blobs-watershed}.png}


\bigskip\hrule\bigskip



\section{Adding Selections to ROI Manager}
\label{\detokenize{basic-segmentation:adding-selections-to-roi-manager}}\begin{enumerate}
\sphinxsetlistlabels{\arabic}{enumi}{enumii}{}{.}%
\item {} 
\sphinxAtStartPar
Press \sphinxcode{\sphinxupquote{T}} to open the ROI manager.

\item {} 
\sphinxAtStartPar
Use \sphinxstylestrong{Edit > Selection > Create Selection} to select all of the blobs based on the current threshold.

\item {} 
\sphinxAtStartPar
In the ROI Manager window, use \sphinxstylestrong{More > Split} to divide the selection into multiple ROIs.
\begin{itemize}
\item {} 
\sphinxAtStartPar
You can click on the regions to see the specific blob it corresponds to.

\item {} 
\sphinxAtStartPar
If needed, the selected regions can be saved in a file and reopened later using \sphinxstylestrong{More > Save}.

\item {} 
\sphinxAtStartPar
One can also add selections to the ROI manager manually, if there are only a few objects to segment or if thresholding segmentation does not work for your sample.

\end{itemize}

\end{enumerate}


\bigskip\hrule\bigskip



\section{Analyze Particles}
\label{\detokenize{basic-segmentation:analyze-particles}}
\sphinxAtStartPar
The \sphinxstylestrong{Analyze Particles} function can be useful for applying size exclusion or removing any blobs that are below our desired size threshold.\\
For example, we can see some smaller red circles in the image that may represent noise rather than a full blob we are interested in.\\
This can be especially useful if we have a biological problem, such as segmenting cells, where we know the cells must be above a certain size.
\begin{enumerate}
\sphinxsetlistlabels{\arabic}{enumi}{enumii}{}{.}%
\item {} 
\sphinxAtStartPar
Open the ROI manager by pressing \sphinxcode{\sphinxupquote{T}}.
\begin{itemize}
\item {} 
\sphinxAtStartPar
If there are previous ROI selections that you would like to remove:
\begin{itemize}
\item {} 
\sphinxAtStartPar
Click on the first one, hold \sphinxcode{\sphinxupquote{Shift}}, then click on the last one.

\item {} 
\sphinxAtStartPar
Press \sphinxcode{\sphinxupquote{Delete}} to clear them.

\end{itemize}

\end{itemize}

\item {} 
\sphinxAtStartPar
Use \sphinxstylestrong{Analyze > Analyze Particles} to open the function window.
\begin{itemize}
\item {} 
\sphinxAtStartPar
\sphinxstylestrong{Size} can be used to determine what blobs are included. For example, specify \sphinxcode{\sphinxupquote{"30\sphinxhyphen{}Infinity"}} to remove smaller blobs.
\begin{itemize}
\item {} 
\sphinxAtStartPar
For many biological applications, it’s best to know the size of the pixels to understand how much area is excluded, and whether it corresponds to the expected size based on the experiment.

\end{itemize}

\item {} 
\sphinxAtStartPar
Ensure the \sphinxstylestrong{Add to Manager} box is selected to add the ROIs to the ROI manager.

\item {} 
\sphinxAtStartPar
For the \sphinxstylestrong{Show} dropdown, select \sphinxstylestrong{Masks} to display the following:

\end{itemize}

\end{enumerate}

\sphinxAtStartPar
\sphinxincludegraphics{{analyze-particles}.png}
\sphinxincludegraphics{{blobs-analyzed}.png}

\begin{sphinxVerbatim}[commandchars=\\\{\}]
 \PYGZgt{} Notice that some of the blobs that are present in the binarized image, like the one next to number 29, are not included in the mask.
\end{sphinxVerbatim}


\bigskip\hrule\bigskip



\section{Masks for Measurement}
\label{\detokenize{basic-segmentation:masks-for-measurement}}\begin{enumerate}
\sphinxsetlistlabels{\arabic}{enumi}{enumii}{}{.}%
\item {} 
\sphinxAtStartPar
Once you have a mask or the desired image, use \sphinxstylestrong{Edit > Selection > Create Selection} to add the regions to the ROI manager.

\item {} 
\sphinxAtStartPar
After selection, you can also use \sphinxstylestrong{Edit > Selection > Create Mask} to make a mask of the desired areas if the image is not already binarized.

\item {} 
\sphinxAtStartPar
To copy the ROI selection onto the original image:
\begin{itemize}
\item {} 
\sphinxAtStartPar
Click on the mask while the selection is open.

\item {} 
\sphinxAtStartPar
Then click on the original image and press \sphinxstylestrong{Shift + E} or use \sphinxstylestrong{Edit > Selection > Restore Selection}.

\end{itemize}

\end{enumerate}


\section[]{\lowercase{\sphinxincludegraphics}{{blobs-masked}.png}}
\label{\detokenize{basic-segmentation:id1}}

\section{Making Measurements}
\label{\detokenize{basic-segmentation:making-measurements}}\begin{enumerate}
\sphinxsetlistlabels{\arabic}{enumi}{enumii}{}{.}%
\item {} 
\sphinxAtStartPar
To set the desired measurements to be collected, use \sphinxstylestrong{Analyze > Set Measurements}.\\
This window will determine what results are displayed or saved from the image.

\end{enumerate}

\sphinxAtStartPar
\sphinxincludegraphics{{set-measurements}.png}
\begin{enumerate}
\sphinxsetlistlabels{\arabic}{enumi}{enumii}{}{.}%
\setcounter{enumi}{1}
\item {} 
\sphinxAtStartPar
To create the measurements:
\begin{itemize}
\item {} 
\sphinxAtStartPar
Use \sphinxstylestrong{Analyze > Measure} or \sphinxstylestrong{Ctrl + M}, while the desired image is active.

\item {} 
\sphinxAtStartPar
This may produce one measurement for the whole image if an ROI is not active.

\item {} 
\sphinxAtStartPar
To measure all ROIs at once, use \sphinxstylestrong{Measure} in the ROI Manager.
\begin{itemize}
\item {} 
\sphinxAtStartPar
The checkboxes for \sphinxstylestrong{Show All} and \sphinxstylestrong{Labels} can be useful to see what regions are being measured.

\end{itemize}

\item {} 
\sphinxAtStartPar
If you want, use \sphinxstylestrong{Edit > Selection > Make Inverse}.
\begin{itemize}
\item {} 
\sphinxAtStartPar
This will now select the entire background.

\item {} 
\sphinxAtStartPar
You can measure again to get the value for the background.

\end{itemize}

\end{itemize}

\end{enumerate}

\sphinxstepscope


\chapter{Basic Registration}
\label{\detokenize{basic-registration:basic-registration}}\label{\detokenize{basic-registration::doc}}

\section{Stitching Images}
\label{\detokenize{basic-registration:stitching-images}}
\sphinxAtStartPar
There are multiple ways to stitch images using the \sphinxstylestrong{Grid/Collection Stitching} plugin, including with metadata or file position.


\begin{enumerate}
\sphinxsetlistlabels{\arabic}{enumi}{enumii}{}{.}%
\item {} 
\sphinxAtStartPar
Open the \sphinxstylestrong{Grid/Collection Stitching} plugin.\\
In this example, use \sphinxstylestrong{Grid: snake by rows} with the order of \sphinxstylestrong{Right \& Down}.

\end{enumerate}

\sphinxAtStartPar
\sphinxincludegraphics{{grid-stitcher}.png}
\begin{enumerate}
\sphinxsetlistlabels{\arabic}{enumi}{enumii}{}{.}%
\setcounter{enumi}{1}
\item {} 
\sphinxAtStartPar
In the dialogue box, specify the following parameters:
\begin{itemize}
\item {} 
\sphinxAtStartPar
Grid size: \sphinxstylestrong{\sphinxcode{\sphinxupquote{x = 3}}, \sphinxcode{\sphinxupquote{y = 3}}}

\item {} 
\sphinxAtStartPar
Tile overlap {[}\%{]}: \sphinxcode{\sphinxupquote{0}}\\
(This is a known value — the image we are stitching was part of a larger image. In an experimental context, this may be part of the image acquisition settings.)

\item {} 
\sphinxAtStartPar
First file index: \sphinxcode{\sphinxupquote{1}}

\item {} 
\sphinxAtStartPar
Directory: point to the \sphinxcode{\sphinxupquote{Leaf\_stitch}} folder

\item {} 
\sphinxAtStartPar
File name: \sphinxcode{\sphinxupquote{leaf\sphinxhyphen{}\{i\}.tif}}
\begin{itemize}
\item {} 
\sphinxAtStartPar
\sphinxcode{\sphinxupquote{\{i\}}} specifies where in the filename to iterate through values.

\item {} 
\sphinxAtStartPar
If this is written as \sphinxcode{\sphinxupquote{\{ii\}}}, there will be an error because the first file is \sphinxcode{\sphinxupquote{1}}, not \sphinxcode{\sphinxupquote{01}}.

\end{itemize}

\item {} 
\sphinxAtStartPar
Uncheck the \sphinxstylestrong{Compute overlap} box, since the overlap value is known.

\end{itemize}

\end{enumerate}

\sphinxAtStartPar
\sphinxincludegraphics{{fused-leaf}.png}
\begin{enumerate}
\sphinxsetlistlabels{\arabic}{enumi}{enumii}{}{.}%
\setcounter{enumi}{2}
\item {} 
\sphinxAtStartPar
Also try re\sphinxhyphen{}running the plugin with an overlap value of 10\%.
\begin{itemize}
\item {} 
\sphinxAtStartPar
Notice how there are errors near the borders of the leaf and on the ruler (especially the faded “9” value), but errors at the center of the leaf may be harder to spot.

\item {} 
\sphinxAtStartPar
This is why knowing the expected overlap value is important, as spotting errors in experimental data may be difficult.

\item {} 
\sphinxAtStartPar
In many cases, using stitching information from the image metadata is more helpful, because the position information comes from the microscope.

\end{itemize}

\end{enumerate}

\sphinxstepscope


\chapter{Scripting}
\label{\detokenize{basic-scripting:scripting}}\label{\detokenize{basic-scripting::doc}}
\sphinxAtStartPar
Scripts can be generated in a variety of ways, including the \sphinxstylestrong{Macro Recorder}.\\
For functions included in the ImageJ Macro language, please see:\\
\sphinxurl{https://imagej.net/ij/developer/macro/functions.html}

\sphinxAtStartPar
More general scripting info:\\
\sphinxurl{https://imagej.net/scripting/}


\bigskip\hrule\bigskip



\section{Macro Recorder}
\label{\detokenize{basic-scripting:macro-recorder}}
\sphinxAtStartPar
\sphinxurl{https://imagej.net/scripting/macro}
\begin{enumerate}
\sphinxsetlistlabels{\arabic}{enumi}{enumii}{}{.}%
\item {} 
\sphinxAtStartPar
Use \sphinxstylestrong{Plugins > Macros > Record} to start recording simple steps and begin generating code.
Note that when clicking around in FIJI, you may generate unnecessary or overly specific commands, which means macros generated through the recorder will typically require some editing before use on other images.

\end{enumerate}

\sphinxAtStartPar
For example:
\begin{itemize}
\item {} 
\sphinxAtStartPar
Open \sphinxstylestrong{Blobs} via \sphinxstylestrong{File > Open Samples > Blobs}

\item {} 
\sphinxAtStartPar
Draw a circle or rectangle using the tools

\item {} 
\sphinxAtStartPar
Run \sphinxstylestrong{Process > Find Edges}

\end{itemize}

\sphinxAtStartPar
\sphinxincludegraphics{{blobs-macro}.png}

\sphinxAtStartPar
More example scripts:\\
\sphinxurl{https://imagej.net/ij/macros/}


\subsection{Why does the recorder not show a response for all plugins?}
\label{\detokenize{basic-scripting:why-does-the-recorder-not-show-a-response-for-all-plugins}}
\sphinxAtStartPar
Not every developer has made the plugins compatible with the macro recorder, which
may cause various bugs or a lack of response from the recorder. In some cases, a script may be
used to generate similar functionality, but other times this may just be a limitation of the plugin.
The \sphinxhref{https://forum.image.sc}{Image.sc forum} would be a good resource for help if you are able to provide enough
information about the error and the analysis goals.


\section[]{\lowercase{\sphinxincludegraphics}{{helen-blobs-circular}.png}}
\label{\detokenize{basic-scripting:id1}}

\section{Batch Processing}
\label{\detokenize{basic-scripting:batch-processing}}\begin{enumerate}
\sphinxsetlistlabels{\arabic}{enumi}{enumii}{}{.}%
\item {} 
\sphinxAtStartPar
Batch processing can be done directly via the macro script editor using a provided template:
\sphinxhref{https://imagej.net/scripting/batch\#option-2---script-template}{https://imagej.net/scripting/batch\#option\sphinxhyphen{}2—script\sphinxhyphen{}template}

\item {} 
\sphinxAtStartPar
Open the script editor: \sphinxstylestrong{Plugins > New > Macro}

\item {} 
\sphinxAtStartPar
The template can be accessed in the script editor by using  \sphinxstylestrong{Templates > ImageJ 1.x > Batch > Process Folder (ImageJ Macro)}

\end{enumerate}

\sphinxAtStartPar
This template contains an existing loop to process each file in the folder
using the “processFolder” and “processFile” functions. It requires the user
to specify the input and output directory as well as the file information.
\begin{enumerate}
\sphinxsetlistlabels{\arabic}{enumi}{enumii}{}{.}%
\item {} 
\sphinxAtStartPar
When you are inserting the macro recorder script into the bottom
function, remember to open and save the file using the \sphinxstylestrong{\sphinxcode{\sphinxupquote{open()}}} and \sphinxstylestrong{\sphinxcode{\sphinxupquote{saveAs()}}} functions.

\end{enumerate}
\begin{itemize}
\item {} 
\sphinxAtStartPar
The benefits of this is that the files can be more specifically iterated if
needed with minor adjustments to the code. For example, try changing \sphinxcode{\sphinxupquote{i++}} to \sphinxcode{\sphinxupquote{i = i + 2}} to process every other file.

\end{itemize}

\sphinxAtStartPar
\sphinxincludegraphics{{helen-macro}.png}

\sphinxstepscope


\chapter{Deconvolution}
\label{\detokenize{deconvolution:deconvolution}}\label{\detokenize{deconvolution::doc}}
\sphinxAtStartPar
Deconvolution is a method of denoising a microscopy image using information about the optical
setup. In this case, we estimate a point spread function for the deconvolution using the user
input parameters of numerical aperture, resolution, emission wavelength, and refractive index.

\sphinxAtStartPar
A Fiji macro developed by \sphinxstylestrong{Ed Evans} is available here:\\
\sphinxurl{https://github.com/elevans/fiji-scripts/blob/main/imagej2/deconvolution/decon.groovy}

\sphinxAtStartPar
More background on the algorithm (Richardson\sphinxhyphen{}Lucy + Total Variation Regularization):\\
\sphinxurl{https://doi.org/10.1002/jemt.20294}
\begin{enumerate}
\sphinxsetlistlabels{\arabic}{enumi}{enumii}{}{.}%
\item {} 
\sphinxAtStartPar
Copy\sphinxhyphen{}paste the code into the script editor, or open the \sphinxcode{\sphinxupquote{.groovy}} file. \sphinxstylestrong{Change the script language to Groovy}, then run the script on the selected image.

\end{enumerate}

\sphinxAtStartPar
\sphinxincludegraphics{{ed-deconv-macro}.png}
\begin{itemize}
\item {} 
\sphinxAtStartPar
A dialog will prompt for the experimental parameters. Enter the known parameters from the experiment
(or for the demo use the default values).

\end{itemize}

\sphinxAtStartPar
\sphinxincludegraphics{{ed-deconv-prompt}.png}
\begin{quote}

\sphinxAtStartPar
Example: Try with \sphinxstylestrong{Cell Colony} (FIJI Sample) and default parameters to see a demonstration.\\
You’ll notice sharper edges and reduced background noise — even though the parameters may not match the real imaging conditions.
\end{quote}

\sphinxAtStartPar
\sphinxincludegraphics{{ed-deconv-psf}.png}

\sphinxstepscope


\chapter{Background Subtraction – Gaussian Filters}
\label{\detokenize{background-subtraction:background-subtraction-gaussian-filters}}\label{\detokenize{background-subtraction::doc}}
\sphinxAtStartPar
Performing background subtraction with a Gaussian filter can be an effective method of resolving
objects from a noisy background. More info here:
\sphinxurl{https://bioimagebook.github.io/chapters/2-processing/4-filters/filters.html\#gaussian-filters}

\sphinxAtStartPar
\sphinxincludegraphics{{gaussian-filter-example}.png}

\sphinxAtStartPar
Background Subtraction \sphinxhyphen{} Gaussian filters
Performing background subtraction with a gaussian filter can be an effective method of resolving
objects from a noisy background. More info here:
\sphinxurl{https://bioimagebook.github.io/chapters/2-processing/4-filters/filters.html\#gaussian-filters}
\begin{enumerate}
\sphinxsetlistlabels{\arabic}{enumi}{enumii}{}{.}%
\item {} 
\sphinxAtStartPar
Use \sphinxstylestrong{File > Open Samples > Neuron (5 channel)} to open the starting image. We only
want one channel to work with, so right click and \sphinxstylestrong{duplicate channel 4}.

\end{enumerate}
\begin{itemize}
\item {} 
\sphinxAtStartPar
Try a variety of thresholds on this image, including the triangle method. Here we
will try to improve this threshold by separating out the smaller point\sphinxhyphen{}like
structures.

\end{itemize}
\begin{enumerate}
\sphinxsetlistlabels{\arabic}{enumi}{enumii}{}{.}%
\setcounter{enumi}{1}
\item {} 
\sphinxAtStartPar
Open the gaussian subtraction script found here, developed by Ed Evans:

\sphinxAtStartPar
\sphinxurl{https://github.com/elevans/fiji-scripts/blob/main/imagej2/filters/ijo\_gaussian\_subtraction.py}

\end{enumerate}

\sphinxAtStartPar
(Download raw file in GitHub.)

\sphinxAtStartPar
Open this in Fiji to open the Macro Editor, or open the Macro Editor via \sphinxstylestrong{Plugins > New > Macro}

\sphinxAtStartPar
This script is written in python, so be sure to change the macro language to
python before running, otherwise an error will be produced:

\sphinxAtStartPar
\sphinxincludegraphics{{macro-language}.png}
\begin{enumerate}
\sphinxsetlistlabels{\arabic}{enumi}{enumii}{}{.}%
\setcounter{enumi}{2}
\item {} 
\sphinxAtStartPar
With the single channel image highlighted, run the script. You should be prompted to
input a sigma value for the gaussian filter. Larger values will blur out larger and larger
objects. For this example, use 20 and then use 1. Feel free to try a variety of values to
see how the results change.

\end{enumerate}

\sphinxAtStartPar
\sphinxincludegraphics{{gaussian-sub-1}.png}
\begin{enumerate}
\sphinxsetlistlabels{\arabic}{enumi}{enumii}{}{.}%
\setcounter{enumi}{3}
\item {} 
\sphinxAtStartPar
After running the script, we now see that some structures are highlighted a bit differently
in result (V) (notice how the cell body is harder to see)
\begin{itemize}
\item {} 
\sphinxAtStartPar
Some structures are highlighted differently.

\item {} 
\sphinxAtStartPar
The cell body may be less visible, while smaller axons or dots stand out more.

\end{itemize}

\end{enumerate}

\sphinxAtStartPar
\sphinxincludegraphics{{gaussian-sub-2}.png}
\begin{enumerate}
\sphinxsetlistlabels{\arabic}{enumi}{enumii}{}{.}%
\setcounter{enumi}{4}
\item {} 
\sphinxAtStartPar
After running the script, we now see that some structures are highlighted a bit differently
in \sphinxcode{\sphinxupquote{result (V)}} (notice how the cell body is harder to see):

\end{enumerate}

\sphinxAtStartPar
\sphinxincludegraphics{{gaussian-sub-3}.png}
\sphinxincludegraphics{{gaussian-sub-4}.png}
\begin{enumerate}
\sphinxsetlistlabels{\arabic}{enumi}{enumii}{}{.}%
\setcounter{enumi}{5}
\item {} 
\sphinxAtStartPar
If we decide this mask is sufficient for our data, we can then apply the mask to the output
image and merge the display.
a. With the output image still selected, choose \sphinxstylestrong{Apply} on the threshold window,
while making sure the dark background box is selected. It then will ask you to
convert the output to an 8\sphinxhyphen{}bit mask.
i. The areas of interest should now have values of 255 while the
background has a value of \sphinxcode{\sphinxupquote{0}}. If this is inverted, it is likely that the dark
background box in the threshold window was not selected, you can simply
use \sphinxstylestrong{Edit > Invert} to switch the values.
b. To overlay the mask on the original image, we can make a composite. This
requires that the original image is also converted to 8\sphinxhyphen{}bit for display. Use \sphinxstylestrong{Image >
Type > 8\sphinxhyphen{}bit}

\end{enumerate}

\sphinxAtStartPar
c. To make the composite, use \sphinxstylestrong{Image > Color > Merge Channels}. Here, I will set
the original channel duplicate to C4 (gray) and the mask to C2 (green). This
should procedure the following composite image, that can be used for display:

\sphinxAtStartPar
\sphinxincludegraphics{{gaussian-sub-composite-20}.png}

\sphinxAtStartPar
d. As an additional display, we could use \sphinxstylestrong{Edit > Selection > Create Selection} on the mask image,
then \sphinxstylestrong{Edit > Selection > Restore Selection} on the original image to get outlines of the mask.

\sphinxAtStartPar
If we use a sigma value of 1 and repeat the process, we can see there a further differences in
the cell body and other structures, although this seems to also pick up more noise:

\sphinxAtStartPar
\sphinxincludegraphics{{gaussian-sub-composite-1}.png}

\sphinxstepscope


\chapter{TrackMate – Example from documentation}
\label{\detokenize{trackmate:trackmate-example-from-documentation}}\label{\detokenize{trackmate::doc}}
\sphinxAtStartPar
Ershov, D., Phan, M.\sphinxhyphen{}S., Pylvänäinen, J. W., Rigaud, S. U., Le Blanc, L., Charles\sphinxhyphen{}Orszag, A., …\\
Tinevez, J.\sphinxhyphen{}Y. (2022). TrackMate 7: integrating state\sphinxhyphen{}of\sphinxhyphen{}the\sphinxhyphen{}art segmentation algorithms into tracking pipelines. \sphinxstyleemphasis{Nature Methods, 19}(7), 829–832. doi:\sphinxhref{https://doi.org/10.1038/s41592-022-01507-1}{10.1038/s41592\sphinxhyphen{}022\sphinxhyphen{}01507\sphinxhyphen{}1}

\sphinxAtStartPar
TrackMate documentation and tutorials:\\
\sphinxurl{https://imagej.net/plugins/trackmate/}\\
TrackMate manual:\\
\sphinxurl{https://imagej.net/media/plugins/trackmate/trackmate-manual.pdf}

\sphinxAtStartPar
\sphinxstyleemphasis{The following demo can be accessed here:}
\sphinxurl{https://napari.imagej.net/en/latest/examples/trackmate.html}
\begin{enumerate}
\sphinxsetlistlabels{\arabic}{enumi}{enumii}{}{.}%
\item {} 
\sphinxAtStartPar
Open the \sphinxcode{\sphinxupquote{trackmate\_example\_data.tif}}.

\item {} 
\sphinxAtStartPar
Run the TrackMate plugin (\sphinxstylestrong{Plugins > Tracking > TrackMate}). Select “Next” as we don’t need to make any changes. The settings in this window allow you to crop the original data as needed.

\end{enumerate}

\sphinxAtStartPar
\sphinxincludegraphics{{trackmate-window-1}.png}
\begin{enumerate}
\sphinxsetlistlabels{\arabic}{enumi}{enumii}{}{.}%
\setcounter{enumi}{2}
\item {} 
\sphinxAtStartPar
Use the Laplacian of Gaussian (LoG) detector. The plugin provides various algorithms to help process the image, also including the difference of gaussian and hessian detectors.

\item {} 
\sphinxAtStartPar
Enter 17 as the estimated object diameter and 0 as a quality threshold, “Next.”

\end{enumerate}

\sphinxAtStartPar
\sphinxincludegraphics{{trackmate-window-2}.png}
\begin{enumerate}
\sphinxsetlistlabels{\arabic}{enumi}{enumii}{}{.}%
\setcounter{enumi}{4}
\item {} 
\sphinxAtStartPar
The next page provides a summary. Select “Next.”

\item {} 
\sphinxAtStartPar
For Initial Thresholding, verify you see 1496 spots selected, and select “Next.”

\end{enumerate}

\sphinxAtStartPar
\sphinxincludegraphics{{trackmate-window-3}.png}
\begin{enumerate}
\sphinxsetlistlabels{\arabic}{enumi}{enumii}{}{.}%
\setcounter{enumi}{6}
\item {} 
\sphinxAtStartPar
Select “Next” as we do not need a filter. If a filter was required, we could use the “Plus” button to add a filter and adjust the selection metric.

\end{enumerate}

\sphinxAtStartPar
\sphinxincludegraphics{{trackmate-window-4}.png}
\begin{enumerate}
\sphinxsetlistlabels{\arabic}{enumi}{enumii}{}{.}%
\setcounter{enumi}{7}
\item {} 
\sphinxAtStartPar
Use the “Simple LAP tracker”, then select “Next.” TrackMate also provides a variety of other methods.

\end{enumerate}

\sphinxAtStartPar
\sphinxincludegraphics{{trackmate-window-5}.png}
\begin{enumerate}
\sphinxsetlistlabels{\arabic}{enumi}{enumii}{}{.}%
\setcounter{enumi}{8}
\item {} 
\sphinxAtStartPar
Then we set the parameters for the tracker. The linking max distance will be 8.3 microns, the gap\sphinxhyphen{}closing max distance is 5 microns, and the gap\sphinxhyphen{}closing max frame gap is 2 micron. Select “Next.”

\end{enumerate}

\sphinxAtStartPar
\sphinxincludegraphics{{trackmate-window-6}.png}
\begin{enumerate}
\sphinxsetlistlabels{\arabic}{enumi}{enumii}{}{.}%
\setcounter{enumi}{9}
\item {} 
\sphinxAtStartPar
A summary page is displayed. You can scroll through the image and see the proposed tracks for each cell. Select “Next.”

\item {} 
\sphinxAtStartPar
Select “Next” three more times as we will not filter any of the tracks and do not need to edit the display options.

\item {} 
\sphinxAtStartPar
Select “Execute” for the final image, which can be run over all of the frames.

\end{enumerate}

\sphinxAtStartPar
\sphinxincludegraphics{{trackmate-window-7}.png}

\sphinxstepscope


\chapter{Labkit Segmentation}
\label{\detokenize{labkit:labkit-segmentation}}\label{\detokenize{labkit::doc}}
\sphinxAtStartPar
\sphinxurl{https://imagej.net/plugins/labkit/}

\sphinxAtStartPar
Arzt, M., Deschamps, J., Schmied, C., Pietzsch, T., Schmidt, D., Tomancak, P., … Jug, F.\\
(2022). LABKIT: Labeling and Segmentation Toolkit for Big Image Data. \sphinxstyleemphasis{Frontiers in Computer Science, 4}. doi:\sphinxhref{https://doi.org/10.3389/fcomp.2022.777728}{10.3389/fcomp.2022.777728}

\sphinxAtStartPar
Labkit uses a pixel classifier to create image segmentation. This requires labeling the foreground and background of an image.
\begin{enumerate}
\sphinxsetlistlabels{\arabic}{enumi}{enumii}{}{.}%
\item {} 
\sphinxAtStartPar
Open Blobs with \sphinxstylestrong{File > Open Samples > Blobs}. Then open Labkit using \sphinxstylestrong{Plugins > Labkit > Open Current Image with Labkit}.

\end{enumerate}

\sphinxAtStartPar
\sphinxincludegraphics{{labkit-1}.png}
\begin{enumerate}
\sphinxsetlistlabels{\arabic}{enumi}{enumii}{}{.}%
\setcounter{enumi}{1}
\item {} 
\sphinxAtStartPar
The Labkit window will automatically load the selected image. Next, use the pencil tool to draw a line on a few pixels that are background (blue) and foreground (red).

\end{enumerate}

\sphinxAtStartPar
\sphinxincludegraphics{{labkit-2}.png}
\begin{enumerate}
\sphinxsetlistlabels{\arabic}{enumi}{enumii}{}{.}%
\setcounter{enumi}{2}
\item {} 
\sphinxAtStartPar
Now, we can run the pixel classifier to segment the image. Use the “play” button for the specific classifier within the segmentation part of the window.

\end{enumerate}

\sphinxAtStartPar
\sphinxincludegraphics{{labkit-3}.png}
\sphinxincludegraphics{{labkit-4}.png}
\begin{enumerate}
\sphinxsetlistlabels{\arabic}{enumi}{enumii}{}{.}%
\setcounter{enumi}{3}
\item {} 
\sphinxAtStartPar
This should produce a classified image similar to the one shown below with blue as the background and red as the object of interest.

\end{enumerate}

\sphinxAtStartPar
\sphinxincludegraphics{{labkit-5}.png}
\begin{enumerate}
\sphinxsetlistlabels{\arabic}{enumi}{enumii}{}{.}%
\setcounter{enumi}{4}
\item {} 
\sphinxAtStartPar
The classifier can be saved and applied to multiple images. Click on the settings button on the classifier to save either the classifier or the image results.

\end{enumerate}

\sphinxAtStartPar
\sphinxincludegraphics{{labkit-6}.png}

\sphinxstepscope


\chapter{Other Resources}
\label{\detokenize{extra-resources:other-resources}}\label{\detokenize{extra-resources::doc}}\begin{itemize}
\item {} 
\sphinxAtStartPar
\sphinxstylestrong{Introduction to Bioimage Analysis (Pete Bankhead)}:\\
\sphinxurl{https://bioimagebook.github.io/README.html}

\item {} 
\sphinxAtStartPar
\sphinxstylestrong{ImageJ User Guide (Ferreira and Rasband)}:\\
\sphinxurl{https://imagej.net/ij/docs/guide/146.html}

\item {} 
\sphinxAtStartPar
\sphinxstylestrong{Introduction and Segmentation in FIJI (YouTube)}:\\
\sphinxhref{https://www.youtube.com/watch?app=desktop\&amp;v=CZExS\_mkGsQ}{https://www.youtube.com/watch?app=desktop\&v=CZExS\_mkGsQ}

\item {} 
\sphinxAtStartPar
\sphinxstylestrong{ImageJ Tutorials}:\\
\sphinxurl{https://imagej.net/imaging/}

\item {} 
\sphinxAtStartPar
\sphinxstylestrong{COBA: Center for Open Bioimage Analysis (YouTube)}:\\
\sphinxurl{https://www.youtube.com/@cobacenterforopenbioimagea1864/featured}

\item {} 
\sphinxAtStartPar
\sphinxstylestrong{Scientific Figure Making with Fiji and Inkscape (Jan Brocher)}:\\
\sphinxurl{https://www.youtube.com/watch?v=F6ll37NOgXc}

\item {} 
\sphinxAtStartPar
\sphinxstylestrong{ImageJ Documentation: Built\sphinxhyphen{}in Macro Functions}:\\
\sphinxurl{https://wsr.imagej.net/developer/macro/functions.html}

\item {} 
\sphinxAtStartPar
\sphinxstylestrong{ImageJ Macro Cheatsheet (Robert Haase)}:\\
\sphinxurl{https://github.com/BiAPoL/imagej-macro-cheat-sheet/blob/master/ImageJ\_macro\_cheatsheet.pdf}

\item {} 
\sphinxAtStartPar
\sphinxstylestrong{ImageJ Documentation: ImageJ Ops} – allows one to use other libraries (ex. OpenCV) natively in ImageJ:
\sphinxurl{https://imagej.net/libs/imagej-ops/}

\item {} 
\sphinxAtStartPar
\sphinxstylestrong{TrackMate Information}:\\
\sphinxurl{https://imagej.net/plugins/trackmate/}

\item {} 
\sphinxAtStartPar
\sphinxstylestrong{StarDist Information}:\\
\sphinxurl{https://github.com/stardist/stardist}

\item {} 
\sphinxAtStartPar
\sphinxstylestrong{Labkit Information}:\\
\sphinxurl{https://github.com/juglab/labkit-ui}

\item {} 
\sphinxAtStartPar
\sphinxstylestrong{I2K Conference (YouTube)}:\\
\sphinxurl{https://www.youtube.com/@I2KConference}

\end{itemize}







\renewcommand{\indexname}{Index}
\printindex
\end{document}